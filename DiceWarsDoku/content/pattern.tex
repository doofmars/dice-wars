\chapter{Verwendete Design-Pattern}  

RandomGenerator (Singleton)\\ 

Den Zufallsgenerator, der wahre Zufallszahlen mit Hilfe von der Random.org API generiert, 
haben wir als Singleton implementiert. Von dem RandomGenerator soll es nur eine Instanz geben. 
Man ruft von au�en nur die Methode "rollTheDice" auf (diese generiert dann, 
je nachdem ob man eine Internetverbindung hat oder nicht, 
richtige oder Pseudo-Zufallszahlen und gibt deren Summe zur�ck). 
Mehrere Instanzen eines RandomGenerators anzulegen w�re daher nicht sinnvoll, 
da die Methode "rollTheDice" nie mehrmals parallel aufgerufen werden muss.\\ \\

Builder-Pattern\\ 

Das Builder-Pattern wird bei uns verwendet um das Spielfeld zu erzeugen. Wir
verwenden zweimal das Pattern. Einmal wie erw�hnt f�r das "Board" und einmal
f�r das "BoardState". Das Builder-Pattern ist daf�r da, um die Erstellung
komplexer Objekte zu vereinfachen.\\ \\

DTO-Pattern\\ 

Das DTO-Pattern verwenden wir um Daten zu sammeln und diese dann in die
Datenbank zu speichern. So sammeln wir an verschiedenen Stellen im Code Daten
und in der letzten Form speichern wir diese in die Datenbank.\\
Das DTO-Pattern ist daf�r da, primitive Datentypen abzuspeichern bzw.
einzusammeln.

\newpage
Observer-Pattern\\

Da wir auf dem Spielfeld die W�rfelanzal anzeigen und diese sich st�ndig �ndert,
haben wir hier das Observer-Pattern implementiert. Nachdem jeder Spieler seine
Z�ge gemacht hat, werden W�rfel neu vergeben bzw. auf die Vorhandenen addiert.
Jetzt muss die GUI und alle dazugeh�rigen Objekte aktualisiert werden.\\ 
Durch die Observer-Funktionalit�t lassen sich solche Probleme leicht l�sen.\\ \\

Somit haben wir vier Design-Pattern implementiert. (Gefordert waren hier drei
Pattern)

